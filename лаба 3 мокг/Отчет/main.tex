\documentclass[14pt, a4paper]{extreport}

\usepackage{susu}

% ====================================================================================================
\begin{document}

\author{Гаврилова~А.Ю.}
\group{212}
\task{2}
\maketitle

% ====================================================================================================
\chapter{Задание}

\begin{enumerate}

	\item
	Разработать программу для построения пересечения двух прямоугольников и его закраски. Предполагается, что стороны прямоугольников параллельны координатным осям. Для задания положения и размеров прямоугольников использовать генератор псевдослучайных чисел. Интерфейс программы должен содержать следующие элементы управления:
	\begin{itemize}
		\item построение фигур;
		\item построение решения;
		\item сохранение результата в файл;
		\item выход из программы.
	\end{itemize}

\end{enumerate}


% ====================================================================================================
\chapter{Математическая модель}
Координаты $x$ и $y$ -- псевдослучайные числа в множестве точек $A_i(x_i,y_i)$, при $i$=1,...$n$\\
При это заданы координаты первого прямоугольника:($x_{min},y_{min} $),($x_{max},y_{max} $), и второго прямоугольника: ($fillbar.x0, fillbar.y0$),($x,y$)\\
Расположение прямоугольников на поле выбирается случайным образом, путем построения 4 точек.\\
Далее, определяется область пересечения прямоугольников проверкой нахождения выбранных точек в области другого прямоугольника. Закраска искомой области происходит по точкам:\\($fillbar.x0, fillbar.y0$), $(fillbar.x1, fillbar.y1)$



% ====================================================================================================
\chapter{Текст программы}

\noindent Файл main.cpp
\lstinputlisting{source/main.cpp}
\pagebreak
\hrulefill

\noindent Файл task.h
\lstinputlisting{source/task.h}
\hrulefill

\noindent Файл task.cpp
\lstinputlisting{source/task.cpp}
\hrulefill

\noindent Файл control.h
\lstinputlisting{source/control.h}
\hrulefill

\noindent Файл control.cpp
\lstinputlisting{source/control.cpp}

% ====================================================================================================
\chapter{Результат работы}

\begin{figure}[h!]
	\centering
	\includegraphics[width = 12cm]{image/output}
  \caption{Результат выполнения программы }
\end{figure}

\begin{figure}[h!]
	\centering
	\includegraphics[width = 12cm]{image/output1}
  \caption{Результат выполнения программы }
\end{figure}



% ====================================================================================================
\end{document}